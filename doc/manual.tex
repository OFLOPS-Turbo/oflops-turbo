\documentclass{book}

\usepackage[pdftex]{graphicx}
\usepackage{listings}
\usepackage{url}

\title{OFLOPS: User manual}
\date{\today}

\begin{document}

\maketitle

\chapter{Introduction}
\section{}

\section{Why OFLOPS?}

\section{OFLOPS design}

\chapter{Installation}

\section{Source code}

The source code of OFLOPS is availiable only throught git repository.
In order to get a copy of the code, users are required to have installed 
the git command line tools. the source code can be downloaded using the 
following command \footnote{Currently, latest release of the source code is not 
  on the git repo, but user can get an early access view of the code through 
    an e-mail to cr409@cl.cam.ac.uk}: 
\begin{quote}
git clone git://gitosis.stanford.edu/oflops.git
\end{quote}

In order to build the code, OFLOPS requires two additional git packages. 
Firstly, it requires the source code of the NetFPGA packet 
generator c library. Because this library is also stored in a git repository, 
we have integrate the code as a git submodule and users can download it
issuing the following command:

\begin{quote}
git submodule init \&\& git submodule update
\end{quote}

Secondly, users are required to have a version of the openflow header file.
By default the OFLOPS build configuration expects the OpenFlow reference
implementation to reside in the directory where OFLOPS code. In order
for a user to get access to the source code, he can issues the following
commands:

\begin{quote}
git clone git://gitosis.stanford.edu/openflow \\
git checkout -b release/1.0.0 remotes/origin/release/1.0.0
\end{quote}

For the compilation of the OFLOPS code it is required to install a
series of libraries, namely pcap~\footnote{http://www.tcpdump.org/}, 
GNU Scientific Library~\footnote{http://www.gnu.org/software/gsl/}, 
libconfig~\footnote{http://www.hyperrealm.com/libconfig/} 
and net-snmp~\footnote{http://www.net-snmp.org/} Additionally, if the users 
require doxygen documentation of the soure code, they can get automaticaly 
if they install the relevant package. The OFLOPS build configuration script 
is configured to check for the presence of the 
doxygen software package and generate the appropriate makefile if it is 
present. In the case of a centos 5.3 distribution all the required libraries 
can be installed using the following command.

\begin{quote}
yum install gcc automake autoconf libtool libpcap-devel net-snmp-devel
doxygen gsl-devel\\
wget http://www.hyperrealm.com/libconfig/libconfig-1.4.7.tar.gz\\
tar -xvzf libconfig-1.4.7.tar.gz \\
cd libconfig-1.4.7 \\
./configure \\
make \&\& make install 
\end{quote}

In order to build the source code readers must first build the c library of the
netfpga oflops packet generator. Assuming that the user current working 
directory is the OFLOPS directory, then the following sequence of commands 
should be issued:

\begin{quote}
cd netfpga-packet-generator-c-library/ \\
./autogen.sh \\
./configure \\
make \\
cd .. \\
./boot.sh \\
./configure \\
make \\
\end{quote}

In case the user wants to use an alternavite location for the OpenFlow reference
implementation, users may use the flag --with-openflow-src-dir= in the
configuration script and define an alternative location for the default
\emph{`pwd`/../openflow}. 

\section{NetFPGA support}

OFLOPS platform provides is fully integrated with the NetFPGA platform.
We provide together with the OFLOPS source code a binary bitfile of hardware
design that enable hardware level packet generation and capturing. 
Both of these features are
important for the precision of the measurement, because software packet generation 
and capture is, on one hand, limited in 
precision by the OS scheduler and, on the other hand, it is constraint by the 
bandwidth of the PCI bus. In order to use the hardware extesion, the
user needs to acquire and install a NetFPGA card and its base software system.
We encourage readers to visit the NetFPGA website for further
details~\footnote{\url{http://www.netfpga.org}}. Once the NetFPGA plaform is
configure on the host machine, users only need to download the appropriate btfile to the FPGA
circuit and modify accordingly the configuration file of OFLOPS. In order to
download the bitfile, the following command is sufficient. 

\begin{quote}
nf2\_download netfpga-packet-generator-c-library/oflops\_packet\_generator.bit
\end{quote}

\chapter{Running OFLOPS}

\section{Introduction}

OFLOPS builds its testing modules as shared libraries. Modules are 
loaded at run time using the dlopen functionality of libc. In order to
configure the test modules loaded during run-time and configure the
parameters of a test, the OFLOPS tool uses a simple syxtax language based on
the libconfig format. In this chapter we describe the configuration parameter
that the OFLOPS platform provides.

\section{OFLOPS configuration files}
\label{oflops-config-lang}

The OFLOPS configuration file is based upon the libconfig file syntax, with 
a strict scheme for the configuration parameters. A set of example configuration 
files can be found in directory sample\_config/. The OFLOPS configuration file 
consists of a unique group parameter, named oflops, which contains a set 
of values that group the configuration parameters based on the generic aspect
of the functionality that they define. Furtermore, the oflops group contains a 
scalar parameters that defined which packet generation library will be used. The value is called 
\emph{traffic\_generation} and has 3 possible values: 
\begin{itemize}
  \item 1, use userspace packet generator.
  \item 2, user pktgen kenrel generator~\footnote{http://www.linuxfoundation.org/collaborate/workgroups/networking/pktgen}.
  \item 3, use NetFPGA-based packet generator. 
\end{itemize}
Regarding the rest of the configuration parameters, they are grouped in 
the following group parameters: 

\subsection{control channel configuration}
For the control section of the configuration file OFLOPS parses the following parameter:
\begin{itemize}
\item \emph{control\_dev}: The name of the network device used for the control channel of 
the controller. This device shouldn't be used at the same time as a data channel. 
\item \emph{control\_port}: The TCP port on which the OFLOPS controller will listen on.
\item \emph{snmp\_addr}: The IP address of the SNMP server of the OpenFlow switch.
\item \emph{cpu\_mib}: A list of MIB entries for the CPU cores of the switch, semicolon 
separated.
\item \emph{in\_mib}: The MIB entry of the packet counter of the input queue of the control 
channel switch port. 
\item \emph{out\_mib}: The MIB entry of the packet counter of the output queue of the control 
channel switch port.
\end{itemize} 

\subsection{data channel configuration}

In order to configure the data channels used by an experiment, OFLOPS 
configuration file contains an array parameter called data which consist 
of variable number device description. The described network device will
be used by the OFLOPS testing module in order to generate traffic on the data
plane of the switch. 

\begin{itemize}
\item \emph{dev}: The name of the network device used by the channel. Each 
network device can only be used once per experiment. 
\item \emph{port\_num}: The ID of the port of the OpenFlow switch to which 
the network device is connected. This is required so that the tesing modules
will know which port ID to use on a flow modification message. 
\item \emph{in\_snmp\_mib}: The SNMP MIB entry for the packet counter of the 
input queue of the switch port. 
\item \emph{out\_snmp\_mib}: The SNMP MIB entry for the packet counter of the
output queue of the switch port. 
\item \emph{type}: The library used to capture packets over the specific data
channel. The possible values are \emph{nf2}, for NetFPGA packet capturing, and 
\emph{pcap}, for libpcap-based packet capturing. 
\end{itemize}

\subsection{module configuration}

Finally the configuration file contains an entry with a variable number
of module configuration entries. The module entries can be used to locate 
and initialiaze appropriately a testing module. Specifically, for each entry
OFLOPS accepts the following parameters. 

\begin{itemize}
\item \emph{path}: The path for the loadable library of the module. 
\item \emph{param}: A space separate array of module parameters with their 
repsective value. The list of parameters is specific for each module and is 
described in a great extend in Chapter~\ref{oflops-modules}
\end{itemize}

\subsection{Example configuration script}

In order to provide an example of an oflops configuration file we present the
following configuration script. In this script we configure the control channel
to use the device eth1 and open a tcp socket on port 6633, while it will query
the snmp server on IP 192.168.1.2 in order get feedback from the switch. In this
experiment we are configuring also two data channels on device nf2c0 and nf2c1
on which we will be using the the netfpga packet capturing functionality.
Finally, for the experiment we dictate the OFLOPS to load the shared library
file libopenflow\_action\_delay.so, to which we pass a series of parameters. 

\lstset{language=Java,numbers=left,frame=single,title=oflops.cfg}
\begin{lstlisting}[frame=single]                % Start your code-block
oflops: {
control: {
           control_dev = "eth1";
           control_port = 6633;
           snmp_addr = "192.168.1.2";
           cpu_mib="1.3.6.1.2.1.25.3.3.1.2.768;1.3.6.1.2.1.25.3.3.1.2.769";  
           in_mib="1.3.6.1.2.1.2.2.1.11.2";
           out_mib="1.3.6.1.2.1.2.2.1.17.2";
           snmp_community = "public";
         };

         data = ({
             dev="nf2c0";
             port_num=1;
             in_snmp_mib="1.3.6.1.2.1.2.2.1.11.3";
             out_snmp_mib="1.3.6.1.2.1.2.2.1.17.3";
             type="nf2";
             },{
             dev="nf2c1";
             port_num=2;
             in_snmp_mib="1.3.6.1.2.1.2.2.1.11.4";
             out_snmp_mib="1.3.6.1.2.1.2.2.1.17.4";
             type="nf2";
             });

         traffic_generator = 3;
         dump_control_channel=0;


module: ({
         path="libopenflow_action_delay.so";
         param="flows=10 data_rate=10 pkt_size=150 ";
         });
};


\end{lstlisting}

% 
% \section{OFLOPS h/w design}
% \label{oflops-hw}
% 
% The OFLOPS hw design is based on the Stanford Packet Generator design. Using this design as reference we 
% extend some of the functionality, in order to cover the needs of our design.
% Specifically, the desing changes that we performed are the following: 
% 
% Reader should be aware, that the h/w design developed as part of the OFLOPS tool has 

\chapter{OFLOPS Modules}
\label{oflops-modules}

\section{Introduction}

In this section we will provide an in depth description of the testing module
that we have developed  so far using the OFLOPS platform. For each module we describe
the scenario it implements, the configuration parameters that it provides and
the output information that it provide. 

\section{openflow\_action\_delay}        

This module is developed in order to measure the impact of the implementation 
of an action on the data plane of a switch. It has been observed in 
some OpenFlow implementation that some of the actions cannot be applied at 
line rate or they are not implemented at all. In order to test the switching
performance of an action, the module initialiazed the switch flow table with a
single flow that simply outputs packet to a specific output port. This action is
kept in the cache for one minute, while the module generate matching traffic and
measures the switching delay. After one minute the module sends a flow
modification that adds on the action list of the flow a set of user defined
actions, and modifies the output port while it continues to measure the switching 
delay incrurred by the switch. By comparing how the number between the two
phases, users can understand whether the action is implemented at line rate and
whether the action incurs any significant delay. 

The module provide the following set of parameters that the user can define:
\begin{itemize}
\item emph{pkt\_size}: This parameter can be used to control the length of
packets of the measurement probe, measured in bytes. Thus, together with the 
rate parameter, it allows indirectly to adjust the packet throughput of the 
experiment. (default 1500 bytes)
\item \emph{data\_rate}: The rate of the measurement probe measured in Mbps.
(default 10Mbps) 
\item \emph{table}: The parameter define wheter the inserted flow will be 
a wildcard(value of 1) or exact match(value of 0).  (default 1)
\item \emph{action}:  A comma seperate string of entries of the format
action\_id/action\_value. E.g. a value of `b/1010,0/2` defines that the action
will modify the tcp/udp port of the matching packet to a value of 1010 and the
packet will be outputed on port 2. (default no action)
\end{itemize}

The module provide output for the result of the experiment on the log file. 
Specifically, it outputs 2 lines, one for each expeeriment phase, with the average, median 
and variance of the RTT delay of the probe and the packet
loss, calculated through SNMP. 

\section{openflow\_flow\_stats \& openflow\_aggr\_flow\_stats}

These modules are used to understand the possible impact of the flow stats
extraction functionality of the OpenFlow protocol on the control and data plane of the
switch. The two modules follow a similar experimental setup, but they differ on
the type of the flow stat requests that they send. The first module uses the
flow\_stats\_request OpenFlow mesage type, while the latter uses the
aggr\_flow\_stats\_request. 
The scenario for this experiment is as follows: the flow table is
initialized with a set of flows for which OFLOPS generates relevant traffic. 
The measurement traffic consists of 2 probes: a \emph{constant probe}, that
matches a single flow, and a \emph{variable probe}, that uses at
random any of the inserted flows. The experiment runs for 1 minute and at 
the same time the controller queries the switch for its flow statistics over 
specific intervals. The module stores statistics for the measurement probe, as
well as, the reply time for the flow statistics and the CPU utilization. 

The module provides the following configuration parameters:
\begin{itemize}
\item \emph{flows}: The total number of unique flow that the module will
initialize the flow table of the switch. (default 128)
\item \emph{query}: The number of unique flows that the module will query the 
switch in each flow request. Because the matching method of the module is based 
on the netmask field of the matching field. (default 128)
\item \emph{pkt\_size}:  This parameter can be used to control the length of the
packets of the measurement probe. It allows indirectly to adjust the packet
throughput of the experiment.
\item \emph{data\_rate}: The rate, in Mbps, of the variable probe. (default
    10Mbps)
\item \emph{probe\_rate}: The rate, in Mbps, of the constant probe. (default 10Mbps)
\item \emph{query\_delay}: The delay, in microseconds, between the different 
stats requests. (default 10000 usec) 
\item \emph{print}: A parameter that defines whether the module will output full
per packet details of the measurement probes. If this value is set to 1, then
the module will print on a file called "measure.log" for each capture packet a
comma separated record with the timestamps of the generation and capture times of the
packet, the packet id, the port at which the packet was captured and the flow id 
of the flow that was used in order to switch the packet. (default 0)
\item \emph{table}: This parameter controls wheter the inserted flow will be
a wildcard(value of 1) or exact match(value of 0). (default 0)
\end{itemize}

The output of the module is printed in the log file. The module prints, after the
end of the experiment, a log line for each flow\_stat\_reply containg timing
information. Additionally, the module prints RTT and loss statistics for each
measurement probe. Finally, the module queries the switch every 10 seconds and
outputs CPU informations in the log file. 

As a future development of this module we are considering the potentiality to
estimate the accuracy of the statistic extraction mechanism. Unfortunately
though, so far we are unable to sunchronize accurately the result of packet
counters of the capturing library, with the results of flow\_stats\_reply 
packets. 

\section{openflow\_packet\_out}

This module is developed to measure the performance of the packet\_out
functionality of the OpenFlow protocol. This message type, allows a
controller to inject packets in the network by explicitly sending the packet
data to the switch and defining the output port of the packet. The module
provide basic measurement on the performance of the OpenFlow implementation
given a set of parameters. During the experiment, a constant flow of packet\_out
message is sent to the switch over
the control channel while the module capture outputed packets on the data plane of the
switch. This way the module can claculate the processing delay incurred to such
message types, as well as, any implication that may occur when the controller
increases it's sending rate. 

The module provides the following configuration parameters:
\begin{itemize}
\item \emph{pkt\_size}:  This parameter can be used to control the length of the
packets of the packet\_out message in bytes. It allows indirectly to adjust the packet
throughput of the experiment. (default 1500 bytes)
\item \emph{probe\_snd\_interval}: This parameter controls the data rate of the 
measurement probe, in Mbps. (default 10Mbps)
\item \emph{print}: This parameter enables the measurement module to print
extended per packet measurement information. The information is printed in log
file. (default 0)
\end{itemize}

The module outputs in the log file of the program SNMP informations from the switch, queried
every 10 seconds, as well as statistics for the RTT and packet loss of the
packets inserted using the packet\_out message type. If the parameter 1 is set
to one, then the module will additionally print per packet timestamps. 

\section{openflow\_packet\_in}

This module is developed to measure the performance of the packet\_in
functionality of the OpenFlow protocol. This message type is generated when a
paqket is received by the switch, that does not match any flow in the flow
table. For this experiment, the module initially removes any flows in the flow table and starts
to generate traffic at a specific rate on one of the ports of the switch. 
In paraller, the module receive packet\_in messages and computer per packet the
processing delay.

The module provides the following configuration parameters:
\begin{itemize}
\item \emph{pkt\_size}:  This parameter can be used to control the length of the
packets of the measurement probe. It allows indirectly to adjust the packet
throughput of the experiment. (default 1500 bytes)
\item \emph{probe\_snd\_interval}: This parameter controls the data rate of the
measurement probe, in Mbps. (default 10Mbps)
\item \emph{print}: This parameter defines if the measurement module prints
extended per packet measurement information. The information is printed in log
file. 
\end{itemize}

The module outputs in the OFLOPS log file SNMP informations from the 
switch, queried every 10 seconds, as well as statistics for the RTT and packet 
loss of the measurement probe, as meausured through the packet\_in messages.

\section{openflow\_reactive}

This module can be used to understand the scalabitity properties, when an
OpenFlow switch is used in a reactive way, similar to the functionality that the NOX
controller provides. In this case, the controller inserts flows only when a
packet\_in event is generated. In order to simulate a usage scenario like 
this, the module initially removes any flows from the switch flow table and starts to generate
measurement traffic. The measurement traffic matches a specific number of
exact match flows and packets are generate in a way that matching flows are
used sequentially. The module logs for each flow the time it received the first
packet, the time it received the packet\_in message and the time a packet of the
flow was received on the data plane. 

The module exports the following parameters:
\begin{itemize}
\item \emph{pkt\_size}: This parameter can be used to control the length of the
packets of the measurement probe. It allows indirectly to adjust the packet
throughput of the experiment. The parameter uses bytes as measurement unit.
\item \emph{probe\_rate}: The rate of the mesaruement probe, measured in Mbps. 
\item \emph{flows}: The number of unique flows that the measurement flows will
generate.
\item \emph{print}:  This parameter enables the measurement module to print
extended per flow measurement information. The information is printed in log
file.
\end{itemize}

The module outputs in the OFLOPS log file the total delay incurred in order to
setup all of the flows of the experiment. In addition, the module polls the 
switch for SNMP packet counters and CPU usage every 10 sec and when the experiment 
completes and save the query result in the log file. FInally, if the print 
parameter is enabled, the module output in a CVS files a set of 
records,  one for each flow, that contain: the flow id and the timestamps of the time the first
packet of the flow was generated, the time when the first packet\_in event of
the flow was received by the controller and the time the first packet of the
received on the data plane. 

\section{openflow\_add\_flow  \& openflow\_mod\_flow}

This set of modules can be used to measure the scalability properties of the flow
table. The modules measure the delay that the switch incurs when the controller
modifes the content of the flow table. The differenece between the two modules is that the first
module considers only flows insertions, while the later focuses on flow updates.
The experimental scenario works as follows: During the initialization of the
experiment a set of flows is inserted in the flow table in order to switch all
of the traffic in the data plane. For the case of flow additions, the module
insert a single wildcarded flow that outputs data to port 3 of the 
experiment, while for flow modification, the module inserts a number of flows in the flow
table equal to the number of flows in the data plane that output matching
packets to port 3. The measurement traffic consists of 2 probes: a \emph{constant probe}, that
matches a single parameter, and a \emph{sequencial probe}, that uses in a round
robbin manner all of the inserted flows. We use two measurement probes
because we are interested to measure the delay to insert both a single flows as
well all of the flow. The experiment keeps the initial content of the flow table 
for 30 seconds, in order for the switch to reach a
steady state, and then sends a series of flow modification and a barrier
request. The module monitors all ports and stores information of the time 
when each new flow modification becomes active and when the barrier reply is
received. 

The module provides the following configuration parameters: 
\begin{itemize}
\item \emph{pkt\_size}: This parameter can be used to control the length of the
packets of the measurement probe. It allows indirectly to adjust the packet
throughput of the experiment. The parameter uses bytes as measurement unit.
The parameter applies for both measurement probes. 
\item \emph{probe\_rate}: The rate of the sequencial probe, measured in Mbps. 
\item \emph{data\_rate}: The rate of the constant probe, measured in Mbps. 
\item \emph{flows}:  The number of unique flows that the module will
update/insert.
\item \emph{table}:  This parameter controls wheter the inserted flow will be
a wildcard(value of 1) or exact match(value of 0). For the wildcard flows, the
module wildcars all of the fields except the destination IP address. 
\item \emph{print}: This parameter enables the measurement module to print
extended per packet measurement information. The information is printed in a
seperate CSV file, named "action\_aggregate.log".
\end{itemize}

The module logs in the OFLOPS log file the timestamps when the flow
modification and barrier request transmittion starts(entry label:
    START\_FLOW\_MOD, BARRIER\_REPLY) 
and ends(entry label: END\_FLOW\_MOD) and when each flow becomes active in the data 
plane(entry label: FLOW\_INSERTED). Additionally, the module prints for each
channel statistics regarding the RTT and packet loss for each data channel. 

\section{openflow\_echo\_delay}

This module can be used to measure the processing delay of the control channel.
Specifically, it generated echo\_request messages for one minute, and calculates the delay
for the switch to reply. The module provides a single configuration 
parameter, \emph{delay}, which define the interrequest delay in microseconds. 

The module outputs in the log file a line for each echo\_reply, that
contains all the required timing information, for offline analysis. The
module also output at the end of the measurement an entry with mean, median and
variance of the reply delay as well as any possible reply loss. 

\section{openflow\_interaction\_test}

This test was developed in order to understand how the flow stats extraction
mechanism may interfere with other control plane operations. In this experiment
we repeat the scenario described in the openflow\_flow\_stats module, but
after 1 minute the module tries to modify the flow table of the switch. In
parallel the module monitors the time it takes for the new flows to become
active on the data plane. 

The module exports the following configuration parameters:
\begin{itemize}
\item \emph{pkt\_size}: This parameter can be used to control the length of the
packets of the measurement probe. It allows indirectly to adjust the packet
throughput of the experiment. The parameter uses bytes as measurement unit.
The parameter applies for both measurement probes.
\item \emph{probe\_rate}: The rate of the sequencial probe, measured in
Mbps.
\item \emph{data\_rate}: The rate of the constant probe, measured in Mbps.
\item \emph{flows}:  The number of unique flows that the module will
update/insert.
\item \emph{table}:  This parameter controls wheter the inserted flow will
be
a wildcard(value of 1) or exact match(value of 0). For the wildcard 
flows, the module wildcars all of the fields except the destination IP address.
\end{itemize}

The module will output on the OFLOPS log file, all the details described in the
openflow\_flow\_stats and openflow\_add\_flow modules. 

\section{openflow\_timer}

With this measurement module, users can quantify the precision of the internal
timers of an OpenFlow switch. The OpenFlow protocol, provides for each flow a
couple of parameters that allows the controller the time for which a flow will be
effective. Once the timers have expired, the switch is obliged to remove the
flow from the flow table and inform the controller with an appropriate 
message, if the flow has a specific flag set. In this experiment, the flow table is
initialized with a default flow with low priority that switches packets to port
2 and in paraller it generate a sequential measurement probe that match a specific
range of flows. After a second of experimental run time, the controller insert a
set of new high priority flows that match the packets of the measurement probe 
and output packets to port 3 of the switch. Each flow has a specific hard timeout 
value assigned to it as well as the OFPFF\_SEND\_FLOW\_REM flag set. The module monitors 
the duration that the flows are active, using the data plane of the switch, as 
well as the duration feedback that it receives from the FLOW\_REM message and
stores them in the log file of OFLOPS. 

The module provides the following configuration parameters:
\begin{itemize}
\item \emph{pkt\_size}: This parameter can be used to control the length of the
packets of the measurement probe. It allows indirectly to adjust the packet
throughput of the experiment. The parameter uses bytes as measurement unit.
The parameter applies for both measurement probes.
\item \emph{data\_rate}: The rate of the constant probe, measured in Mbps.
\item \emph{flows}:  The number of unique flows that the module will
update.
\end{itemize}

\section{snmp\_queue\_delay}

This module is developed in order to measure the switching performance in the
data plane of the switch. This test is important mostly for software
switches, as hardware switches perform switching at line rate using
specialised hardware designs resulting in negiligible processing delay. 
In this experiment the module initializes the flow table with a single
flow that output packets to a specific port. The module generates traffic at
various rates for a specific time period and measure for each packet the switching delay. 
Specifically, the module tests for data rates of 1, 10, 64, 128, 256, 512 and 1000 Mbps.

The module accepts the following configuration parameters:
\begin{itemize}
\item \emph{pkt\_size}: This parameter can be used to control the length of the
packets of the measurement probe. It allows indirectly to adjust the packet
throughput of the experiment. The parameter uses bytes as measurement unit.
\item \emph{duration}: The duration of the measurement period for each probing
rate. The measurement unit is seconds and the default value is 30. 
\end{itemize}

The module will output in the OFLOPS log a set of records which provide for each
probing rate the mean, median and variance of the RTT and the packet loss. 

\section{snmp\_tuple\_test}

In this module we are tesing all possible compinations of the OpenFlow tuple in
order to quantify any possible the impact of the tuple format to the data plane. This test
design occured by the fact that some of the early implementations, did not
support at full extend the OpenFlow tuple. The module uses a single flow entry
and a single flow measurement probe. For each possible tuple compination the
controller send a flow modification with the appropriate flow match and then
sends traffic for a specific period of time. For each test, the module stores in the
log file the RTT and loss statistics of the flow. 

The module provides two possible parameters for this module: \emph{pkt\_size}
and \emph{duration}. The pkt\_size controls the length of the packet of the
meausrement probe. The duration is used to define to the module the duration of
each tuple test. Currently the default value is 10 seconds. 

\section{openflow\_dummy}

This module is a simple reference module that is included in order to examplify
to user all the potential API calls that a module can implement. This module
doesn not implement any functionality and should not be used in any test.

%\section{oflops\_debug}
%
%\section{openflow\_action\_measurement}
%
%A module to measure the delay for a switch to reply to 
%
%\section{openflow\_flow\_dump\_test}
%\section{openflow\_port\_status}
%\section{openflow\_vlan\_mod}
%\section{snmp\_cpu}
%\section{openflow\_action\_install}
%\section{openflow\_path\_delay}

\chapter{Developing OFLOPS modules}

\end{document}
